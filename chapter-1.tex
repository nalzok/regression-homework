\title{《回归分析》作业参考答案 \\ \textbf{第一章}}
\author{
    sword\\
    2017级统计系
    \and
    nalzok\\
    2016级统计系
}
\date{\today}

\documentclass[10pt]{article}
\usepackage[UTF8]{ctex}
% \setCJKmainfont{PingFangSC-Regular}	% I really like Ping Fang, but I'm not sure if it's legal to use.
\setCJKmainfont{SourceHanSansCN-Regular}	% Source Han Sans is open source and free, so let's go for it for the time being.
\usepackage{amssymb,amsmath,amsthm}
\let\oldproofname=\proofname
\renewcommand{\proofname}{\rm\bf{\oldproofname}}
\theoremstyle{definition}
\newtheorem*{solution}{解}
\theoremstyle{definition}
\newtheorem*{definition}{定义}
\usepackage{enumitem}

\begin{document}
\maketitle

\newpage

\section{第一节}

\begin{enumerate}
    \item	% 1
        矩阵$A$为$n$阶实对称矩阵,证明:

        \begin{enumerate}[label=(\roman*), start=3]
            \item
                矩阵$A$有$n$个实特征值(记作$\lambda_1\, ,\lambda_2\, ,\cdots ,\lambda_n$),而且$n$个特征向量可正交化;
            \item
                存在正交阵$P$使得
                \begin{equation*}
                    A = P
                    \begin{pmatrix}
                        \lambda_1 & & \\
                                  & \ddots & \\
                                  & & \lambda_n
                    \end{pmatrix}
                    P' \triangleq P\Lambda P'
                \end{equation*}
            \item
                对任意正整数$s$,有
                \begin{equation*}
                    A^s = P
                    \begin{pmatrix}
                        \lambda_1^s & & \\
                                    & \ddots & \\
                                    & & \lambda_n^s
                    \end{pmatrix}
                    P' \triangleq P\Lambda^s P'
                \end{equation*}
                从而$\mathrm{tr}(A^s)=\sum_{i=1}^{n}\lambda_i^s$;
            \item
                $A$是非奇异的当且仅当$\lambda_i \ne 0\ (\forall i)$,且矩阵$A^{-1}$的特征值为$\lambda_i^{-1}\text{,}i=1\, ,2\, ,\cdots ,n$;
            \item
                矩阵$\mathrm{I}_n+cA$的特征值为$1+c\lambda_i\text{,}i=1\, ,2\, ,\cdots ,n$;
            \item
                半正定矩阵的特征值均非负,从而半正定矩阵的迹非负。
        \end{enumerate}

        \begin{proof}
            \begin{enumerate}[label=(\roman*), start=3]
                \item
                    由代数学基本定理,知矩阵$A$的特征多项式在复数域中有且仅有$n$个根。
                    又由(i)得$A$的复特征值均为实数,故矩阵$A$有$n$个实特征值。
                    下面,借用(iv)的结论,矩阵$A$可相似对角化,则$A$各个特征值的几何重数等于代数重数,所以$A$有$n$个线性无关的特征向量,从而$n$个特征向量可正交化。
                \item
                    对实对称矩阵$A$的阶数$n$作数学归纳法。

                    $n=1$时,$A$本身即为对角阵,再取$P$为一阶单位阵,结论成立。

                    假设任一$n-1$阶实对称矩阵都能正交相似于一对角阵,下面来看$n$阶的情形。

                    设$\lambda_1$为$A$的一个特征值,$\eta_1$为对应的单位特征向量,则$\eta_1$可扩成一个标准正交基$\eta_1\, ,\cdots ,\eta_n$,并记矩阵$T=(\eta_1\, ,\cdots ,\eta_n)$。从而
                    \begin{equation*}
                        AT=T
                        \begin{pmatrix}
                            \lambda & \alpha \\
                            0 & B
                        \end{pmatrix}
                    \end{equation*}
                    又因为
                    \begin{equation*}
                        \begin{pmatrix}
                            \lambda & \alpha \\
                            0 & B
                        \end{pmatrix}'=(T'AT)'=T'AT=
                        \begin{pmatrix}
                            \lambda & \alpha \\
                            0 & B
                        \end{pmatrix}
                    \end{equation*}
                    所以$\alpha = 0$,矩阵$B$为实对称矩阵。由归纳假设,存在$n-1$阶正交阵$Q$,使得$B=Q'\Lambda_1 Q$,其中$\Lambda_1$为对角阵。
                    令$Q_1=\mathrm{diag}(1\, ,Q)$,就有
                    \begin{equation*}
                        T'AT=Q'
                        \begin{pmatrix}
                            \lambda & 0 \\
                            0 & \Lambda_1
                        \end{pmatrix}Q
                    \end{equation*}
                    再取$P=Q_1T'$,即可证得$n$阶实对称矩阵亦正交相似于一对角矩阵。

                    根据数学归纳法,结论成立。
                \item
                    注意到
                    $PP'=\mathrm{I}$,从而$A^s=(P\Lambda P')^s=P\Lambda P'P\Lambda P'\cdots P\Lambda P'=P\Lambda^sP'$。
                    又因为相似矩阵有相同的迹,故$\mathrm{tr}(A^s)=\mathrm{tr}(\Lambda^s)=\sum_{i=1}^{n}\lambda_i^s$。
                \item
                    矩阵$A$非奇异$\Leftrightarrow$矩阵$\Lambda$非奇异$\Leftrightarrow$$\lambda_i \ne 0\ (\forall i)$。

                    又$A^{-1}=(P\Lambda P')^{-1}=P\Lambda^{-1} P'$及相似矩阵有相同的特征值,得矩阵$A^{-1}$的特征值为$\lambda_i^{-1}\text{,}i=1\, ,2\, ,\cdots ,n$。
                \item
                    注意到$\mathrm{I}_n+cA=PP'+P(c\Lambda)P'=P(\mathrm{I}_n+c\Lambda)P'$,与(vi)类似即可得证。
                \item
                    任取非零向量$x$,记向量$y=Px$,则由$P$满秩知$y$亦为非零向量。
                    从而
                    \begin{equation*}
                        x'\Lambda x=x'P'APx=y'Ay \ge 0
                    \end{equation*}
                    将$x$取遍标准单位向量即可证得结论。
            \end{enumerate}
        \end{proof}

    \item 	% 2
        设$A$为实对称矩阵,用拉格朗日乘数法证明:
        \begin{equation*}
            \max_{x'x \ne 0} \frac{x'Ax}{x'x} = \lambda_{\max}(A)\quad \text{及} \quad \min_{x'x \ne 0} \frac{x'Ax}{x'x} = \lambda_{\min}(A)
        \end{equation*}

        \begin{proof}
            不失一般性,考虑$x$为单位向量的情形即可。
            取单位向量$e=(x_1\, ,\cdots ,x_n)'$,则$e$的取值范围为单位球面(从而是有界闭集)。
            易见$e'Ae$为$e$的连续函数,而连续函数在有界闭集上必取得最大、最小值,从而结论中的最值运算是有意义的。

            由$A$实对称,知存在正交阵$P$与对角阵$\Lambda$,使得$A=P'\Lambda P$。
            从而
            \begin{equation*}
                e'Ae=e'P'\Lambda Pe \triangleq y'\Lambda y
            \end{equation*}
            由于正交变换保持距离且为可逆变换,所以向量$y$仍为单位向量且取值范围为单位球面。

            记$y=(y_1\, ,\cdots ,y_n)'$,$\Lambda=\mathrm{diag}(\lambda_1\, ,\cdots ,\lambda_n)$,则
            \begin{equation*}
                y'\Lambda y=\lambda_1y_1^2+\cdots +\lambda_ny_n^2
            \end{equation*}
            令$f(y)=\lambda_1y_1^2+\cdots +\lambda_ny_n^2$,于是问题转化为求解
            \begin{equation*}
                \begin{aligned}
        & \max_{y}f(y) \qquad \mathrm{s.t.} \ \sum_{i=1}^{n}y_i^2=1 \\
        & \min_{y}f(y) \qquad \mathrm{s.t.} \ \sum_{i=1}^{n}y_i^2=1
                \end{aligned}
            \end{equation*}
            应用拉格朗日乘数法,构造拉格朗日函数
            \begin{equation*}
                L(y\, ,\mu)= \lambda_1y_1^2+\cdots +\lambda_ny_n^2-\mu(y_1^2+\cdots +y_n^2-1)
            \end{equation*}
            上式两端对各分量求一阶偏导,得
            \begin{equation*}
                \begin{aligned}
        & \frac{\partial L}{\partial y_i}=2\lambda_iy_i-2\mu y_i=0 \qquad i=1\, ,2\, ,\cdots ,n \\
        & \frac{\partial L}{\partial \mu}=y_1^2+\cdots +y_n^2-1=0
                \end{aligned}
            \end{equation*}
            解得$\mu=\lambda_i \, ,y=\epsilon_i\, ,f(\epsilon_i)=\lambda_i$,其中$\epsilon_i$为第$i$个分量为1,其余分量为0的单位向量。
            由此,遍历$A$的特征值,知结论成立。
        \end{proof}

    \item 	% 3
        设向量$\beta=(\beta_1\, ,\beta_2\, ,\cdots ,\beta_n)'$,证明:

        \begin{enumerate}[label=(\roman*)]
            \item
                对任意$n$维向量$a$,有
                \begin{equation*}
                    \frac{\partial \beta' a}{\partial \beta}=a
                \end{equation*}
                对任意$n$列矩阵$A$,有
                \begin{equation*}
                    \frac{\partial A\beta}{\partial \beta'}=A
                \end{equation*}
            \item
                若$A$为对称矩阵,则
                \begin{equation*}
                    \frac{\partial \beta' A\beta}{\partial \beta}=2A\beta
                \end{equation*}
        \end{enumerate}

        \begin{proof}
            \begin{enumerate}[label=(\roman*)]
                \item
                    设$a=(a_1\, ,a_2\, ,\cdots ,a_n)'$,则$\beta'a=a_1\beta_1+a_2\beta_2+\cdots +a_n\beta_n$。
                    于是
                    \begin{equation*}
                        \frac{\partial \beta' a}{\partial \beta}
                        =\left(\frac{\partial}{\partial \beta_1}\, ,\frac{\partial}{\partial \beta_2}\, ,\cdots ,\frac{\partial}{\partial \beta_n}\right)'(a_1\beta_1+a_2\beta_2+\cdots +a_n\beta_n)
                        =(a_1\, ,a_2\, ,\cdots ,a_n)'
                        =a
                    \end{equation*}
                    又设$A=(b_{ij})_n$,则 \begin{equation*}
                        \begin{aligned}
                            \frac{\partial A\beta}{\partial \beta'} & = \frac{\partial}{\partial \beta'}
                            \begin{pmatrix}
                                b_{11}\beta_1+\cdots +b_{1n}\beta_n \\
                                b_{21}\beta_1+\cdots +b_{2n}\beta_n \\
                                \vdots \\
                                b_{n1}\beta_1+\cdots +b_{nn}\beta_n 
                            \end{pmatrix}\\
                                                                    & = 
                                                                    \begin{pmatrix}
                                                                        \frac{\partial}{\partial \beta_1} \\
                                                                        \frac{\partial}{\partial \beta_2} \\
                                                                        \vdots \\
                                                                        \frac{\partial}{\partial \beta_n}
                                                                    \end{pmatrix} (b_{11}\beta_1+\cdots +b_{1n}\beta_n\, ,\cdots , b_{n1}\beta_1+\cdots +b_{nn}\beta_n ) \\
                                                                    & = A
                        \end{aligned}
                    \end{equation*}
                \item
                    对任意的$n$阶方阵$A=(b_{ij})_n$,有
                    \begin{equation*}
                        \begin{aligned}
                            \frac{\partial \beta' A\beta}{\partial \beta}
        &=\frac{\partial}{\partial \beta}\left(\sum_{i=1}^{n}\sum_{j=1}^{n}b_{ij}\beta_i\beta_j\right) \\
        &=\left(\frac{\partial}{\partial \beta_1}\, ,\frac{\partial}{\partial \beta_2}\, ,\cdots ,\frac{\partial}{\partial \beta_n}\right)'\left(\sum_{i=1}^{n}\sum_{j=1}^{n}b_{ij}\beta_i\beta_j\right) \\
        &=(A+A')\beta
                        \end{aligned}
                    \end{equation*}
                    特别地,当$A$为对称矩阵时,有
                    \begin{equation*}
                        \frac{\partial \beta' A\beta}{\partial \beta}=2A\beta
                    \end{equation*}
            \end{enumerate}
        \end{proof}

    \item	% 4
        设四分块矩阵 \begin{equation*}
            \begin{pmatrix}
                A & B \\
                C & D
            \end{pmatrix}
        \end{equation*}满足$A$与$D-CA^{-1}B$
        均可逆。给出上述四分块矩阵的逆矩阵。

        \begin{solution}
            记矩阵$M=D-CA^{-1}B$,则
            \begin{equation*}
                \begin{pmatrix}
                    A^{-1} & \\
                           & M^{-1}
                \end{pmatrix}
                \begin{pmatrix}
                    \mathrm{I} & -M^{-1}B \\
                               & \mathrm{I}
                \end{pmatrix}
                \begin{pmatrix}
                    I & \\
                    -A^{-1}C & \mathrm{I}
                \end{pmatrix}
                \begin{pmatrix}
                    A & B \\
                    C & D
                \end{pmatrix}=\mathrm{I}
            \end{equation*}
            从而
            \begin{equation*}
                \begin{aligned}
                    \begin{pmatrix}
                        A & B \\
                        C & D
                    \end{pmatrix}^{-1} & =
                    \begin{pmatrix}
                        A^{-1} & \\
                               & M^{-1}
                    \end{pmatrix}
                    \begin{pmatrix}
                        \mathrm{I} & -M^{-1}B \\
                                   & \mathrm{I}
                    \end{pmatrix}
                    \begin{pmatrix}
                        \mathrm{I} & \\
                        -A^{-1}C & \mathrm{I}
                    \end{pmatrix} \\ & =
                    \begin{pmatrix}
                        A^{-1}+A^{-1}M^{-1}BA^{-1}C & -A^{-1}M^{-1}B \\
                        -M^{-1}A^{-1}C & M^{-1}
                    \end{pmatrix}
                \end{aligned}
            \end{equation*}
        \end{solution}

    \item	% 5
        \begin{enumerate}[label=(\roman*)]
            \item 举例说明矩阵$A$的广义逆$A^-$不唯一;
            \item 给出计算矩阵$A$广义逆$A^-$的一般公式。 
        \end{enumerate}

        \begin{solution}
            \begin{enumerate}[label=(\roman*)]
                \item
                    先回顾矩阵广义逆的定义。
                    \begin{definition}
                        设$A$是一个$s\times n$矩阵,矩阵方程$AXA=A$的通解称为$A$的广义逆矩阵,简称为$A$的广义逆,记作$A^{-}$。
                    \end{definition}
                    广义逆矩阵可以不唯一。事实上,对任意非单位阵的幂等阵$A$,有
                    \begin{equation*}
                        \begin{aligned}
        & AAA=A \\
        & A\mathrm{I}A=A
                        \end{aligned}
                    \end{equation*}
                    成立。
                    从而$A$的广义逆可以是$A$和$\mathrm{I}$,不唯一。

                    另一个反例是$A=\begin{pmatrix}0\end{pmatrix}$,此时$A^-$可取$\begin{pmatrix}0\end{pmatrix}$或$\begin{pmatrix}42\end{pmatrix}$或$\begin{pmatrix}-3.14\end{pmatrix}$等。
                    认为退化的矩阵过于特殊的同学也可以自行构造对应的数量矩阵。
                \item
                    记矩阵$A$的秩为$r$,对$A$作满秩分解
                    \begin{equation*}
                        A=P
                        \begin{pmatrix}
                            \mathrm{I}_r & 0 \\
                            0 & 0
                        \end{pmatrix}Q
                    \end{equation*}
                    则矩阵$A$的广义逆的一般形式为
                    \begin{equation*}
                        A^-=Q^{-1}
                        \begin{pmatrix}
                            \mathrm{I}_r & B \\
                            C & D 
                        \end{pmatrix}P^{-1}
                    \end{equation*}
                    其中$B\, ,C\, ,D$分别是任意$r\times (s-r)\, ,(n-r)\times r\, ,(n-r)\times (s-r)$阶矩阵。
                    证明请参阅丘维声著《高等代数》第一版上册251-252页。
            \end{enumerate}
        \end{solution}

    \item	% 6
        对任意$n\times p$阶矩阵$X$,证明:
        \begin{enumerate}[label=(\roman*)]
            \item 无论$(X'X)^-$如何变化,$X(X'X)^-X'$保持不变;\\
            \item$X(X'X)^-X'$是一个从$\mathbb{R}^n$到$\mathcal{M}(X)$的投影矩阵。
        \end{enumerate}

        \begin{proof}
            \begin{enumerate}[label=(\roman*)]
                \item
                    记矩阵$X$的秩为$r$。由线性代数的知识,知将$X$作一系列初等列变换,可得到一个列阶梯矩阵,即
                    \begin{equation*}
                        X=AQ
                    \end{equation*}
                    其中矩阵$A$为列阶梯矩阵(前$r$列为标准单位向量,后$p-r$列为0),$Q$为可逆矩阵。
                    再适当交换矩阵$A$的行,得
                    \begin{equation*}
                        PA=
                        \begin{pmatrix}
                            \mathrm{I}_r & 0 \\
                            0 & 0
                        \end{pmatrix}\triangleq \Lambda
                    \end{equation*}
                    其中,$P$为若干行交换矩阵相乘得到的矩阵(因而是正交阵)。
                    于是,我们得到了矩阵$X$的一个满秩分解$X=P'\Lambda Q$。
                    从而$X'X=Q'\Lambda Q$。
                    由第一章习题一5(2)的结论,$(X'X)^-$的一般形式为
                    \begin{equation*}
                        (X'X)^-=Q^{-1}
                        \begin{pmatrix}
                            \mathrm{I}_r & B \\
                            C & D
                        \end{pmatrix}(Q')^{-1}
                    \end{equation*}
                    其中$B\, ,C\, ,D$分别是任意
                    $r\times (p-r)\, ,(n-r)\times r\, ,(n-r)\times (p-r)$阶矩阵。
                    故
                    \begin{equation*}
                        X(X'X)^-X'=P'\Lambda P
                    \end{equation*}
                    而与$B\, ,C\, ,D$无关,即$X(X'X)^-X'$与$(X'X)^-$的选取无关。
                \item
                    由(1)知$X(X'X)^-X'=P'\Lambda P$,容易验证$X(X'X)^-X'$是一个对称幂等阵,且对$\forall \alpha \in \mathbb{R}^n$,$X(X'X)^-X'\alpha =X((X'X)^-X'\alpha)\in \mathcal{M}(X)$,从而结论成立。
            \end{enumerate}
        \end{proof}

    \item	% 7
        令矩阵$\Sigma =(1-\rho)\mathrm{I}_n+\rho 1_n1_n'$,其中$1_n$为元素全为1的$n$维向量。求$\Sigma$的特征值和对应的特征向量。
        又问$\rho$取何值时$\Sigma$(半)正定?

        \begin{solution}
            由第一章习题一1(vii)的结论,只需考察矩阵$1_n1_n'$的特征值与特征向量即可。
            因为$1_n1_n'$的秩为1且易见行和$n$为其特征值,所以矩阵$1_n1_n'$的特征值为$n$(1重)和0($n-1$重)。
            同时容易求出对应的特征值为$1_n$和$\epsilon_1-\epsilon_j\text{,}j=2\, ,\cdots ,n$。
            从而矩阵$\Sigma$	的特征值为$1+(n-1)\rho$(1重)和$1-\rho$($n-1$重),所对应的特征向量即为上述特征向量。

            由于$\Sigma$是对称矩阵,所以判断$\Sigma$是否(半)正定只需考察其特征值。
            于是容易得出,$\rho=-\frac{1}{n-1}$或$\rho=1$时,$\Sigma$严格半正定;$-\frac{1}{n-1} < \rho <1$时,$\Sigma$正定。

            当然硬要抬杠的话,$n=1$的情况还需要另外讨论。
        \end{solution}

    \item	% 8
        在$p$维向量空间$\mathbb{R}^p$中,超平面是由线性方程组$H_{(A,b)}=\{x\, |\, Ax=b\}$所定义的点集。
        \begin{enumerate}[label=(\roman*)]
            \item 取定点$x_0\in H_{(A,b)}$,证明:$H_{(A,b)}=\{x\, |\, (x-x_0) \perp \mathcal{M}(A')\}$;\\
            \item 若矩阵$A$行满秩,对$\forall x\in \mathbb{R}^p$,计算欧氏距离$\mathrm{d}(x\, ,H_{(A,b)})$。
        \end{enumerate}

        \begin{solution}
            不妨设$A$为$m\times p$阶矩阵。

            \begin{enumerate}[label=(\roman*)]
                \item
                    记集合$S_1=\{x\, |\, Ax=b\}$,$S_2=\{x\, |\, (x-x_0) \perp \mathcal{M}(A')\}$。
                    要证明$S_1$与$S_2$相互包含。

                    一方面,任取$x\in S_1$,有$Ax=b$。
                    对$\forall \alpha \in \mathcal{M}(A')$,存在向量$\beta \in \mathbb{R}^m$,使得$\alpha =A'\beta$。
                    从而
                    \begin{equation*}
                        \begin{aligned}
                            (\alpha \, ,x-x_0) & = \beta'A(x-x_0) \\
                                               & = \beta'(b-b) =0
                        \end{aligned}
                    \end{equation*}
                    即$(x-x_0)\perp \alpha \ (\, \forall \alpha \in \mathcal{M}(A')\, )$。
                    所以$x \in S_2$,$S_1 \subset S_2$。

                    另一方面,任取$y \in S_2$,有$(y-x_0)\perp \mathcal{M}(A')$。
                    取$\mathbb{R}^m$中一个基$\gamma_1\, ,\cdots ,\gamma_m$,令矩阵$T=(\gamma_1\, ,\cdots ,\gamma_m)$,则$T$为$m$阶可逆矩阵。
                    由于
                    \begin{equation*}
                        \begin{aligned}
        & 0=(A'\gamma_1 \, ,y-x_0)=\gamma_1' (Ay-b) \\
        & 0=(A'\gamma_2 \, ,y-x_0)=\gamma_2' (Ay-b) \\
        & \cdots \\
        & 0=(A'\gamma_m \, ,y-x_0)=\gamma_m' (Ay-b)
                        \end{aligned}
                    \end{equation*}
                    从而$T'(Ay-b)=0$,$Ay-b=0$,故$y\in S_1$,$S_2 \subset S_1$。

                    综上,$S_1$与$S_2$相互包含,结论成立。
                    事实上,$H_{(A,b)}$为$\mathbb{R}^p$上的线性流形,$H_{(A,b)}-x_0$为$\mathbb{R}^p$的线性子空间。
                    从$H_{(A,b)}-x_0$的定义上可以看出$H_{(A,b)}-x_0$是$\mathcal{M}(A')$的正交补空间,于是$H_{(A,b)}=\{x\, |\, (x-x_0) \perp \mathcal{M}(A')\}$。
                \item
                    由于矩阵$A$行满秩,所以方程组$Ax=b$必有解,从而$H_{(A,b)}$不为空集。
                    记方程组$Ax=0$的一个基解矩阵为$U$,则$\mathrm{d}(x\, ,H_{(A,b)})$即为向量$x-x_0$到$\mathcal{M}(U)$的距离。
                    若记$y^*$为关于$y$的方程组$U'Uy=U'(x-x_0)$的解,则$\mathrm{d}(x\, ,H_{(A,b)})=\rVert x-x_0-Uy^*\rVert_2$。
            \end{enumerate}
        \end{solution}

    \item	% 9
        设矩阵$A$非奇异,矩阵$D$为一方阵,证明:
        \begin{equation*}
            \begin{vmatrix}
                A & B \\
                C & D
            \end{vmatrix}=|A||D-CA^{-1}B|
        \end{equation*}

        \begin{proof}
            由初等变换,易见
            \begin{equation*}
                \begin{pmatrix}
                    A & B \\
                    C & D
                \end{pmatrix}
                \begin{pmatrix}
                    \mathrm{I} & -A^{-1}B \\
                    0 & \mathrm{I}
                \end{pmatrix}=
                \begin{pmatrix}
                    A & 0 \\
                    C & D-CA^{-1}B
                \end{pmatrix}
            \end{equation*}
            从而结论成立。
        \end{proof}

    \item	% 10
        设矩阵$A\, ,B$均为半正定矩阵(且$A\le B$)。
        \begin{enumerate}[label=(\roman*)]
            \item 证明:$AB^-A\le A$;
            \item$AB^-A$依赖于广义逆的选择吗?
        \end{enumerate}

        \begin{proof}
            \begin{enumerate}[label=(\roman*)]
                \def\labelenumi{(\roman{enumi})}
            \item
                首先,如果没有$A\le B$的条件,
                \begin{equation*}
                    A = \begin{pmatrix} 1 \end{pmatrix},
                    B = \begin{pmatrix} 0 \end{pmatrix},
                    B^- = \begin{pmatrix} 42 \end{pmatrix}.
                \end{equation*}
                即可作为反例。

                加入这个条件之后,$B^-$仍不一定是对称矩阵,从而无法保证$AB^-A$是对称矩阵,偏序关系无从谈起。
            \item
        \end{enumerate}
    \end{proof}

\item	% 11
    举例说明:存在这样的矩阵$\Sigma$,$\Sigma$非半正定,但可以找到正交阵$(P\  Q)$,使得$P'\Sigma P=\Lambda_r\, ,Q'\Sigma Q=0$。 

    \begin{proof}
        令
        \begin{equation*}
            \Sigma=
            \begin{pmatrix}
                1 & 1 \\
                0 & 0
            \end{pmatrix} \qquad P=
            \begin{pmatrix}
                \frac{\sqrt{2}}{2} \\
                \frac{\sqrt{2}}{2}
            \end{pmatrix} \qquad Q=
            \begin{pmatrix}
                -\frac{\sqrt{2}}{2} \\
                \frac{\sqrt{2}}{2}
            \end{pmatrix}
        \end{equation*}
        容易验证,矩阵$\Sigma$的特征值为0和1,$(P\  Q)$为正交阵且$P'\Sigma P=1\, ,Q'\Sigma Q=0$。
    \end{proof}

\item	% 12
    设矩阵$A\, ,B$均为半正定矩阵,且$A\le B$。证明:

    \begin{enumerate}[label=(\roman*)]
        \item$\mathrm{tr}(A) \le \mathrm{tr}(B)$;
        \item 矩阵$A$的最大特征值不大于矩阵$B$的最大特征值;
        \item 矩阵$A$的最小特征值不大于矩阵$B$的最大特征值;
        \item$|A|\le |B|$;
        \item$\mathcal{M}(A)\subset \mathcal{M}(B)$;
        \item 设$P_A\, , P_B$分别是$A\, ,B$的正交投影矩阵,则
            $P_A \le P_B$。
    \end{enumerate}

    \begin{proof}
        \begin{enumerate}[label=(\roman*)]
            \item
                由$A\le B$知$\forall x \in \mathbb{R}^m$,有$x'Ax \le x'Bx$。
                从而将$x$取遍$\epsilon_i\text{,}i=1\, ,2\, ,\cdots ,m$,就得到$A$对角线元素的值均不大于$B$对角线元素的值。
                因此$\mathrm{tr}(A) \le \mathrm{tr}(B)$。
            \item
                易见
                \begin{equation*}
                    \lambda_{\max}(A)=\max_{x'x\ne 0}\frac{x'Ax}{x'x} \le \max_{x'x\ne 0}\frac{x'Bx}{x'x}=\lambda_{\max}(B)
                \end{equation*}
            \item
                同样地,有 \begin{equation*}
                    \lambda_{\min}(A)=\min_{x'x\ne 0}\frac{x'Ax}{x'x} \le \min_{x'x\ne 0}\frac{x'Bx}{x'x}=\lambda_{\min}(B)
                \end{equation*}
            \item
                不妨设矩阵$A\, ,B$均正定。
                由$A\le B$知$\forall x\in \mathbb{R}^m$且$x\ne 0$,有$0 < x'Ax \le x'Bx$。
                从而
                \begin{equation*}
                    \begin{aligned}
                        1 & \ge \frac{x'Ax}{x'Bx} \\
                          & = \frac{y'B^{-\frac 12}AB^{-\frac 12}y}{y'y} \qquad (y=B^{\frac 12}x)
                    \end{aligned}
                \end{equation*}
                因此$\lambda_{\max}(B^{-\frac 12}AB^{-\frac 12})\le 1$。
                又易见矩阵$B^{-\frac 12}AB^{-\frac 12}$为正定矩阵,从而其特征值介于0和1之间(不包含0)。
                因此$|B^{-\frac 12}AB^{-\frac 12}|\le 1$,即$|A|\le |B|$。
            \item
                对$x\in \mathcal{M}(B)^{\perp}$,有
                \begin{equation*}
                    \begin{aligned}
                        Bx=0 & \quad \Rightarrow x'Bx=0 \\
                             & \quad \Rightarrow x'Ax \le x'Bx=0 \\
                             & \quad \Rightarrow x'Ax=0 \\
                             & \quad \Rightarrow (A^{\frac{1}{2}}x)'(A^{\frac{1}{2}}x)=0 \\
                             & \quad \Rightarrow (A^{\frac{1}{2}}x)=0 \\
                             & \quad \Rightarrow Ax=0 \\
                             & \quad \Rightarrow \mathcal{M}(B)^{\perp} \subset \mathcal{M}(A)^{\perp} \\
                    \end{aligned}
                \end{equation*}
                从而$\mathcal{M}(A)\subset \mathcal{M}(B)$。
            \item
                对$\forall \xi \in \mathbb{R}^m$,存在$\xi_1\in \mathcal{M}(A)$和$\xi_2\in \mathcal{M}(A)^{\perp}$使得$\xi=\xi_1+\xi_2$。
                从而
                \begin{equation*}
                    \xi' P_A\xi= \xi' P_AP_A\xi=(P_A\xi)'(P_A\xi)=\xi_1'\xi_1
                \end{equation*}
                于是
                \begin{equation*}
                    \begin{aligned}
                        \xi'P_B\xi & = \xi_1'P_B\xi_1 + \xi_1'P_B\xi_2 +\xi_2'P_B\xi_1 + \xi_2'P_B\xi_2 \\
                                   & = (P_B\xi)'(P_B\xi) + (P_B\xi_1)'\xi_2 + \xi_2'(P_B\xi_1) + \xi_2'P_B\xi_2 \\
                                   & = \xi_1'\xi_1 + \xi_1‘’\xi_2 + \xi_2'\xi_1 + \xi_2'P_B\xi_2 \\
                                   & = \xi_1'\xi_1 + \xi_2'P_B\xi_2 \\
                                   & \ge \xi_1'\xi_1 =\xi' P_A\xi
                    \end{aligned}
                \end{equation*}
                从而$P_A \le P_B$。
        \end{enumerate}
    \end{proof}

\item	% 13
    设矩阵$A\, ,B$均为半正定矩阵,且$\mathcal{M}(A)\subset \mathcal{M}(B)$。
    举例说明$A\le B$未必成立。

    \begin{proof}
        令$A=\mathrm{diag}(1\, ,3\, ,0\, ,0)\ ,B=\mathrm{diag}(1\, ,1\, ,1\, ,0)$,则易见$A$每一列都可由$B$线性表出,从而$\mathcal{M}(A)\subset \mathcal{M}(B)$。
        取向量$x=(0\, ,1\, ,0\, ,0)'$,则$x'Ax>x'Bx$,即$A\le B$不成立。

        事实上,假设命题成立,则根据对称性可知$\mathcal{M}(A) = \mathcal{M}(B) \implies \forall x, x'Ax = x'Bx \implies A = B$。为了构造反例,注意到所有满秩矩阵的列空间均为$\mathbb{R}^n$,因此只需找两个不同的满秩矩阵即可。
    \end{proof}
\end{enumerate}

\end{document}
